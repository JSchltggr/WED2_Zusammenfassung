\documentclass[7pt,landscape,a4paper]{scrartcl}
\usepackage[german]{babel}
\usepackage[utf8]{inputenc}
\usepackage{multicol}
\usepackage[landscape]{geometry}
\usepackage{graphicx}
\usepackage{sectsty}
\usepackage{mwe}
\usepackage{comment}
\usepackage{lipsum}
\usepackage{titlesec}
\usepackage[dvipsnames]{xcolor}
\usepackage{amsmath}
\usepackage{enumitem}
\usepackage{color}
\usepackage[T1]{fontenc}
%format
\geometry{top=0.1cm,left=0.1cm,right=0.1cm,bottom=0.1cm}

%Define Color
\definecolor{dkgreen}{rgb}{0,0.6,0}
\definecolor{gray}{rgb}{0.5,0.5,0.5}
\definecolor{mauve}{rgb}{0.58,0,0.82}
\definecolor{lightlightgrey}{rgb}{0.89,0.89,0.89}
\definecolor{b}{RGB}{0, 120, 255}
\definecolor{lightgray}{rgb}{0.95, 0.95, 0.95}
\definecolor{darkgray}{rgb}{0.4, 0.4, 0.4}
\definecolor{purple}{rgb}{0.65, 0.12, 0.82}
\definecolor{editorGray}{rgb}{0.95, 0.95, 0.95}
\definecolor{editorOcher}{rgb}{1, 0.5, 0} % #FF7F00 -> rgb(239, 169, 0)
\definecolor{editorGreen}{rgb}{0, 0.5, 0} % #007C00 -> rgb(0, 124, 0)
\definecolor{orange}{rgb}{1,0.45,0.13}		
\definecolor{olive}{rgb}{0.17,0.59,0.20}
\definecolor{brown}{rgb}{0.69,0.31,0.31}
\definecolor{purple}{rgb}{0.38,0.18,0.81}
\definecolor{lightblue}{rgb}{0.1,0.57,0.7}
\definecolor{lightred}{rgb}{1,0.4,0.5}

\graphicspath{ {./images/} }

\setkeys{Gin}{width=\linewidth}

% Code Snippets
\usepackage{courier} %% Sets font for listing as Courier.
\usepackage{listings}

% CSS
\lstdefinelanguage{CSS}{
	keywords={color,background-image:,margin,padding,font,weight,display,position,top,left,right,bottom,list,style,border,size,white,space,min,width, transition:, transform:, transition-property, transition-duration, transition-timing-function},	
	sensitive=true,
	morecomment=[l]{//},
	morecomment=[s]{/*}{*/},
	morestring=[b]',
	morestring=[b]",
	alsoletter={:},
	alsodigit={-}
}

% JavaScript
\lstdefinelanguage{JavaScript}{
	morekeywords={typeof, new, true, false, catch, function, return, null, catch, switch, var, if, in, while, do, else, case, break},
	morecomment=[s]{/*}{*/},
	morecomment=[l]//,
	morestring=[b]",
	morestring=[b]'
}

\lstdefinelanguage{HTML5}{
	language=html,
	sensitive=true,	
	alsoletter={<>=-},	
	morecomment=[s]{<!-}{-->},
	tag=[s],
	otherkeywords={
		% General
		>,
		% Standard tags
		<!DOCTYPE,
		</html, <html, <head, <title, </title, <style, </style, <link, </head, <meta, />,
		% body
		</body, <body,
		% footer
		<footer, /footer,
		% Divs
		</div, <div, </div>, 
		% Paragraphs
		</p, <p, </p>,
		% scripts
		</script, <script,
		% More tags...
		<canvas, </canvas, <svg, <rect, <animateTransform, </rect, </svg, <video, <source, <iframe, </iframe, </video, <image, </image, <header, </header, <article, </article, <a, </a, <img, </img, <span, </span, <ul, </ul, <li, </li, <input, <br, <form, <h2, <h1, <h3, <button, <nav, <link, <main, <aside, <style, </style, <label, </label, </form, </button, <figure, </figure, <figcaption, </figcaption, <audio, </audio, </h3
	},
	ndkeywords={
		% General
		=,
		% HTML attributes
		charset=, src=, id=, width=, height=, style=, type=, rel=, href=,
		% SVG attributes
		fill=, attributeName=, begin=, dur=, from=, to=, poster=, controls=, x=, y=, repeatCount=, xlink:href=,
		% properties
		margin:, padding:, background-image:, border:, top:, left:, position:, width:, height:, margin-top:, margin-bottom:, font-size:, line-height:,
		% CSS3 properties
		transform:, -moz-transform:, -webkit-transform:,
		animation:, -webkit-animation:,
		transition:,  transition-duration:, transition-property:, transition-timing-function:,
	}
}

\lstdefinestyle{htmlcssjs} {%
	% General design
	%  backgroundcolor=\color{editorGray},
	basicstyle={\footnotesize\ttfamily},
	aboveskip=0mm,
	belowskip=0mm,   
	frame=none,
	% line-numbers
	%xleftmargin={0.6cm},
	%numbers=left,
	stepnumber=1,
	firstnumber=1,
	numberfirstline=true,	
	% Code design
	identifierstyle=\color{black},
	keywordstyle=\color{blue}\bfseries,
	ndkeywordstyle=\color{editorGreen}\bfseries,
	stringstyle=\color{editorOcher}\ttfamily,
	commentstyle=\color{brown}\ttfamily,
	% Code
	language=HTML5,
	alsolanguage=JavaScript,
	alsodigit={.:;},	
	tabsize=2,
	showtabs=false,
	showspaces=false,
	showstringspaces=false,
	extendedchars=true,
	breaklines=true,
	% German umlauts
	literate=%
	{Ö}{{\"O}}1
	{Ä}{{\"A}}1
	{Ü}{{\"U}}1
	{ß}{{\ss}}1
	{ü}{{\"u}}1
	{ä}{{\"a}}1
	{ö}{{\"o}}1
}


% Define Section Format
\titleformat{name=\section}[block]
{\sffamily\normalsize}
{}
{0pt}
{\colorsection}
\titlespacing*{\section}{0pt}{0.3ex}{0.3ex}

\newcommand{\colorsection}[1]{%
	\colorbox{ForestGreen!40}{\parbox{0.235\textwidth}{\color{black}\thesection\ #1}}}


% Define Subsection Format
\titleformat{name=\subsection}[block]
{\sffamily\small}
{}
{0pt}
{\colorsubsection}
\titlespacing*{\subsection}{0pt}{0.3ex}{0.3ex}

\newcommand{\colorsubsection}[1]{%
	\colorbox{yellow!50}{\parbox{0.235\textwidth}{\color{black}\thesubsection\ #1}}}

% Define Subsubsection Format
\titleformat{name=\subsubsection}[block]
{\sffamily\small}
{}
{0pt}
{\colorsubsubsection}
\titlespacing*{\subsubsection}{0pt}{0.3ex}{0.3ex}

\newcommand{\colorsubsubsection}[1]{%
	\colorbox{BrickRed!40}{\parbox{0.235\textwidth}{\color{black}\thesubsubsection\ #1}}}

% -----------------------------------------------------------------------
\begin{document}
	\setlength{\columnseprule}{0.4pt}
	\footnotesize
\begin{multicols*}{4}
\section{HTML (HyperText Markup Language)}
	\textcolor{b}{\textbf{Auszeichnungssprache} (markup language):}\\
	ist eine maschinenlesbare Sprache für die Gliederung und Formatierung von Texten und anderen Daten.\\
	\textbf{Bsp:} HTML, Markdown, Rich Text Format, LaTeX, XML\\
	$\rightarrow$ Gibt die Definition vor, wie Strukturen definieren werden müssen, damit diese korrekt interpretiert werden.\\
%	\textbf{Parsing:}\\
%	\includegraphics{parsing.png}\\
	\textcolor{b}{\textbf{Tag:}} Markiert Start und Ende eines HTML Elements\\
	\textcolor{b}{\textbf{Element:}} Repräsentiert eine Struktur oder einen Inhalt\\
	\textcolor{b}{\textbf{Attribut/Attribute:}} Definiert Eigenschaft eines Elements\\
	\textcolor{b}{\textbf{Kommentar:}} Start: $<!--$ Ende: $-->$\\
	\textcolor{b}{\textbf{Void Elemente:}} Kein Content. Kein End-Tag. End-Slash erlaubt. z.B: \textbf{br, hr, img, input, link, meta}\\
	\textcolor{b}{\textbf{HTML Entities:}} \textbf{\&} = \&amp; $<$ = \&lt; $>$ = \&gt; $"$ = \&quot;
\subsection{HTML-Struktur \& Encoding}
\begin{lstlisting}[style=htmlcssjs]
<!DOCTYPE html> //Definiert wie der Browser den Text interpretiert und rendert.
<html lang="en"> //Root-Element des Dokuments, 
<head> //Meta-Informationen, wird nicht dargestellt
	<meta charset="UTF-8"> // in ersten 1024 Byte
	<title>Title</title> //Zwingend notwendig
</head>
<body> //Definiert den sichtbaren Inhalt der Webseite
	Hello World
</body> <footer> </footer> </html>
\end{lstlisting}
	\textcolor{b}{\textbf{Warum Charset?}}\\
	Der Browser kann nicht erkennen wie der Byte-Stream zu interpretieren ist. \textbf{Problem:} UTF-8 hat eine variable Länge (8 16 24 oder 32 Bits). \textbf{Ohne Angabe:} Default pro Region wird ausgewählt z.B. ISO 8859-1\\
	$\rightarrow$ File muss mit entsprechendem Encoding gespeichert werden.
	$\rightarrow$ Named entities: Umlaut wie z.B. \&ouml = ö
\subsection{Content-Model}
\begin{minipage}{0.48\linewidth}
	\includegraphics{content_model_2}
\end{minipage}
\begin{minipage}{0.52\linewidth}
	\textbf{Tag Omission:}\\
	Sagt ob ein Tag (Start/End)\\
	- vorhanden sein muss,\\
	- weggelassen werden kann\\
	- oder nicht vorhanden sein darf.
\end{minipage}
\subsection{Wichtigste HTML-Elemente}
\begin{lstlisting}[style=htmlcssjs]
<a href="http://www.google.com">Link</a>
<img src="picture.png" alt="Description">
<input type="password" value="1234">
\end{lstlisting}
	\textbf{Am häufigsten benutzte Tags:}\\
	html, head, body, title, meta, a, div, link, script, img span, ul\\
	li, p, style, input, br, form, h2, h1, iframe, h3, button, footer\\
	header, nav\\
	\textcolor{b}{\textbf{label/input:}}\\
	Vor einem input Element immer ein dazugehörendes label (mit for=$"$input id$"$) definieren. \\
	\includegraphics[width=0.97\linewidth]{html_tags.png}\\
	\textcolor{b}{\textbf{Semantische Elemente (HTML5):}} main, header, footer, nav, figure, figcaption, article, section, aside, details\\
	\textcolor{b}{\textbf{Empfohlene Seiten-Struktur:}}
\begin{lstlisting}[style=htmlcssjs]
<body><header><nav><main><header><nav><aside><footer>
\end{lstlisting}
\section{CSS (Cascading Style Sheets)}
	\textcolor{b}{\textbf{Vorteile:}}
	\begin{itemize}[topsep=0pt, leftmargin=3mm]
		\setlength\itemsep{-0.3em}
		\item Klare Trennung zwischen Struktur und Styling
		\item Ermöglicht es, auf verschiedene Ausgabegrössen zu reagieren.
		\item Unterschiedliche Styles für verschiedene Ausgabemedien möglich
		$\rightarrow$ \textbf{Print, Screen}
		\item Können auf Gruppen von Elementen angewendet werden
		\item Style-Definitionen können in separate Dateien ausgelagert werden
	\end{itemize}
	\textcolor{b}{\textbf{Inline Styling:}} Sytle Deklaration in Tag. (Für Debuggen)
\begin{lstlisting}[style=htmlcssjs]
<span style="background: green; color: pink">
\end{lstlisting}
	\textcolor{b}{\textbf{Style-Regeln:}} style Tag im Head (Deguggen, Produktion)
\begin{lstlisting}[style=htmlcssjs]
<head><style> span{background: green;} </style></head>
\end{lstlisting}
	\textcolor{b}{\textbf{CSS einbinden:}} link Tag im Head (Standard)
\begin{lstlisting}[style=htmlcssjs]
<head><link rel="stylesheet" href="file.css" media="screen"></head>
\end{lstlisting}
	$\rightarrow$ Im Link Tag kann media (print, screen) angegeben werden
	\textcolor{b}{\textbf{Kommentar:}} /* comment */\\
	\textcolor{b}{\textbf{CSS-Deklaration:}} Property:Value; (z.B. color: red;)
\subsection{CSS Selektoren}
	Sind \textbf{Case-Sensitive.} Werte von id und class-Attributen werden klein geschrieben.\\ 
	Der Selektor einer Style-Regel definiert welche HTML-Elements die Style-Deklarationen der Style-Regel angewendet werden.\\
	\begin{tabular}{|l|l}
		\hline
		\textbf{Selektor-Typ} & \textbf{Beispiel}\\
		\hline
		Typ (type) & a \{  \}\\
		\hline
		Universal & * \{  \}\\
		\hline
		ID & \#container \{  \}\\
		\hline
		Klassen (class) & .my-class \{  \}\\
		\hline
		Attribut (attribute) & a$[$ href $]$ \{  \}\\
		\hline
		Pseudo-Element & a::first-line \{  \}\\
		& .question::after \{content: '?';\}\\
		& p::selection \{background: gray;\}\\
		\hline
		Pseudo-Klasse & div:hover \{  \};\\
		& article:not(.my-class) \{  \}\\
		\hline
		Compound & div.box \{alle div mit class box\}\\
		\hline
		Nachfahren (decendant) & p h1 \{irgendwann in p\}\\
		\hline
		Kind (child) & p $>$ h1 \{p ist direkter parent\}\\
		\hline Angrenzende Geschwister & p + a \{a direkt nach p\}\\
		\hline Allgemeine Geschwister & p {\raise.17ex\hbox{$\scriptstyle\mathtt{\sim}$}} a \{irgendwann auf Ebene\}\\
		\hline
	\end{tabular}
\subsection{Developper Tools}
	Können unter Rendering print simulieren\\
	\textbf{Die Dev-Console ermöglicht:}
	\begin{itemize}[topsep=0pt, leftmargin=3mm]
		\setlength\itemsep{-0.3em}
		\item Disable/Re-enable Deklaration(en) einer Regel
		\item Neue Deklaration(en) in einer Regel
		\item Neue Deklaration(en) für das selektierte Element
		\item Neue Regel mit neuen Deklarationen
	\end{itemize}
\subsection{Box-Model}
	Jedes Element besteht aus: \textcolor{b}{Inhalt} (content), \textcolor{b}{Innenabstand} (padding), \textcolor{b}{Rahmen} (border) \& \textcolor{b}{Aussenabstand} (margin)\\
	\begin{minipage}{0.4\linewidth}
		\includegraphics{box_model.png}
	\end{minipage}
	\begin{minipage}{0.6\linewidth}
	\textbf{Grösse von Element:}\\
	Content+Padding+Border+Margin\\
	\textbf{Background:}\\
	Auf Content + padding angewendet\\
	\textbf{Zentrieren:}\\
	margin-left: auto; margin-right: auto;
\end{minipage}
	\textcolor{b}{\textbf{Eigenschaften Border:}}
	\begin{itemize}[topsep=0pt, leftmargin=3mm]
		\setlength\itemsep{-0.3em}
		\item Breite des Rahmens
		\item Jede Kante kann einzel formatiert werden
		\item Abrunden möglich
		\item Bild als rahmen-Hintergrund verwendbar
		\item Border wird der Grösse des Elements hinzugefügt/addiert
	\end{itemize}
	\textcolor{b}{\textbf{Eigenschaften Outline:}}
	\begin{itemize}[topsep=0pt, leftmargin=3mm]
		\setlength\itemsep{-0.3em}
		\item Wird \textbf{nicht} der Grösse des Elements hinzugefügt
		\item Gewisse Elemente ignorieren den Border (z.B. Checkbox); Outline funktioniert aber
		\item Viel weniger Optionen als Border
	\end{itemize}
\subsection{Display, Position, Float, Kaskade, Einheiten}
	\textcolor{b}{\textbf{Display:}}\\
	\textbf{block:} Füllt die ganze Zeile. \textbf{inline-block:} Nimmt nur den notwendigen Platz ein. \textbf{inline:} Ignoriert width \& height. \textbf{none:} Entfernt Element. \textbf{visibility:hidden oder opacity: 0:} Erzeugt WhiteSpace (Platz wird reserviert)\\
	\textcolor{b}{\textbf{Position:}}\\
	\textbf{absolute:} Element überlagert. \textbf{fixed:} immer am gleichem Ort. \textbf{relative:} Platz im Fluss bleibt reserviert. \textbf{static:} Standard-Positionierung im Fluss\\
	\textcolor{b}{\textbf{Float:}}\\
	Um Text um ein Bild fliessen zu lassen. (float:left; float:right;)\\
	\textcolor{b}{\textbf{Kaskade (Cascading):}}\\
	Ist ein Algorithmus, um Style-Quellen zu verschmelzen.
	\begin{itemize}[topsep=0pt, leftmargin=3mm]
		\setlength\itemsep{-0.3em}
		\item Deklarationen im Browser-Stylesheet (kleinste Priorität)
		\item Deklarationen des Benutzers
		\item Deklarationen des Autors
		\item Wichtige Deklarationen (mit !important) des Autors
		\item Wichtige Deklarationen (mit !important) des Benutzers
	\end{itemize}
	\textcolor{b}{\textbf{Einheiten:}}\\
	\textbf{px:} Basis-Einheit. \textbf{em:} Relativ zu Parent Schriftgrösse. \textbf{rem:} Relativ zu Root-Element (html).
\subsection{Tabellen, Formulare, Bilder \& Medien}
	\begin{minipage}{0.53\linewidth}
		\textcolor{b}{\textbf{Table:}}\\
		\\
		\includegraphics{table.png}
	\end{minipage}
	\begin{minipage}{0.48\linewidth}
		Table besteht aus:
		\begin{itemize}[topsep=0pt, leftmargin=3mm]
			\setlength\itemsep{-0.3em}
			\item Table <table>
			\item Caption <caption>
			\item Column Group <colgroup>
			\item Column <col>
			\item Table Head <thead>
			\item Table Body <tbody>
			\item Table Footer <tfoot>
			\item Table Row <tr>
			\item Table Data Cell <td>
			\item Table Header Cell <th>
		\end{itemize}
	\end{minipage}
	\textcolor{b}{\textbf{Form:}}
\begin{lstlisting}[style=htmlcssjs]
<form action="http://localhost:5002/submit" method="get">
	<label for="message"><span>Message: </span>
		<input id="message" name="message" type="text" placeholder="Your Message Here" required>
	</label>
	<button>Send Data</button>
</form>
\end{lstlisting}
	\textbf{Wichtige form Attribute:}\\
	\textcolor{b}{action:} URL zu der die Daten geschickt werden\\
	\textcolor{b}{method} \textbf{get:} Daten werden in der URL/Query-String gesendet, \textbf{post:} Daten werden im Request-Body gesendet.
	\textbf{Wichtigste Form-Sub-Elemente:}\\
	label: Anschrift\\
	fieldset: Gruppierung von Elementen (Sub Element \textcolor{b}{legend})\\
	select: Auswahl (Sub-Element \textcolor{b}{option})\\
	datalist: Vordefinierte Antwort-Optionen\\
	textarea: Mehrzeilige Texte\\
	\textbf{Wichtigste Input-Typen:} button, checkbox, date, email, hidden, month, number, password, radio, range, number, text, week\\
	\textbf{Pseudoklassen:} :invalid, :valid, :focus, :placeholder-shown, :default, :in-range, :disabled, :enabled\\
	\textbf{Pseudo-Element:} ::placeholder\\
	\textcolor{b}{\textbf{Bilder \& Medien}}
\begin{lstlisting}[style=htmlcssjs]
// Bilder
<figure> <img src="./cat.jpg" alt="Elvis" >
	<figcaption>Elvis the HSR Cat</figcaption> </figure>
//Audio
<audio src="./song.ogg" controls></audio>
//Video
<video src="./example.mp4" controls poster="./bild.jpg">
	<a href="./example.mp4">Download video</a></video>
\end{lstlisting}
\section{JavaScript (ECMAScript)}
\begin{lstlisting}[style=htmlcssjs]
<script> alert("TEST"); </script> //Head
<script src="myFile.js"> </script> //Body
<button onclick="alert('hi hsr')"> //Body
\end{lstlisting}
	\textcolor{b}{\textbf{Primitive Typen:}} string, number, boolean, null, undefined, symbol $\rightarrow$ \textbf{Compared by value}\\
	\textcolor{b}{\textbf{Objekte:}} Plain Objekts, Arrays, Regular Expressions, Functions $\rightarrow$ \textbf{Copmared by reference}
\subsection{PRIMITIVES}
	\textcolor{b}{\textbf{Falsy/Truthy:}}\\
	\textbf{False:} false, 0 (zero), $"$$"$ (empty string), null, undefined, NaN\\
	\textbf{True:} Alles andere: true, "0" (string), $"$false$"$ (string), [], \{\} etc.\\
	\textcolor{b}{\textbf{Numbers:}}\\
	Sind immer Floats (Gleitkommazahl).\\
	\textbf{Spezialfall:} NaN (0/0) $\rightarrow$ isNaN() zum prüfen, Infinity (x/0 oder sehr hohe Zahl, kann auch negativ sein)\\
	\textbf{Jeder Wert kann in Typ Number umgewandelt werden:}\\
	+(true)==1, +(false)==0, Number(true)==1,\\
	Number(null)==0, Number($"$abc")=>NaN, +("123")==123\\
	parseInt($"$string")/parseFloat($"$string"): Bis zum ersten Fehler 
	\textcolor{b}{\textbf{String}}\\
	length,	slice(), trim(), includes, indexOf.\\
	String Typ wird mit Backquotes umschlossen. Innerhalb von \$\{...\} wird JavaScript interpretiert. z.B.
\begin{lstlisting}[style=htmlcssjs]
console.log(`Mein Name ist: ${name}`);
\end{lstlisting}
	\textcolor{b}{\textbf{Rechnen:}}\\
	Punkt vor Strich!! String + Value = Value + String = String\\
	\textcolor{b}{\textbf{Vergleichen:}}\\
	=== verhindert Typenumwandlung. Für Objekte nicht nötig.
	Variable ist \textcolor{b}{undefined}, wenn nicht definiert, oder initialisiert.\\
	Variable ist \textcolor{b}{null}, wenn ein Objekt erwartet wird.\\
	\textbf{Achtung:} undefined == null => true\\
	\textcolor{b}{\textbf{Arrays:}}\\
	Haben keine fixe Länge, Index beginnt bei 0.
\begin{lstlisting}[style=htmlcssjs]
const arr = ["a","b","c"]; //über arr Iterieren
//Klassisch:
for (let i=0; i < arr.length; ++i) {
	console.log("for", arr[i]); }
//For-In:
for(let x in arr) {
	console.log("for in", x + ":" + arr[x]); }
//For-Of
for (let y of arr) {
	console.log("for of", y); }
//For-Each:
arr.forEach((elem,index))=>console.log(index+":"+elem);
\end{lstlisting}
	\textcolor{b}{\textbf{Funktionen:}}\\
	Können in \textcolor{b}{Variable} gespeichert werden, als \textcolor{b}{Parameter} mitgegeben werden, von Funktionen \textcolor{b}{zurückgegeben} werden.\\
	Haben \textcolor{b}{offene Parameterliste} (mehr oder weniger als deklariert)\\
	Besitzen \textcolor{b}{Properties}, erzeugen einen eigenen \textcolor{b}{Scope}. 
\begin{lstlisting}[style=htmlcssjs]
//Funktionen können definiert werden
function hallo(){
	console.log("Hallo"); } hallo();
//Funktionen können einer Variable zugewiesen werden
const hallo2 = function() {
	console.log("Hallo2"); }; hallo2();
//Funktionen können einer Variable zugewiesen werden
const foo = hallo; foo();
\end{lstlisting}
\section{DOM (Document Object Model)}
	\textcolor{b}{\textbf{DOM:}}\\
	Ist eine Programmier-Schnittstellendefinition für Dokumente (XML/HTML). $\rightarrow$ Repräsentiert das HTML-Dokument als \textcolor{b}{Baumstruktur}. Jeder Node im Baum ist ein \textbf{Objekt} welches ein Stück vom Dokument repräsentiert. $\rightarrow$ Das DOM definiert Methoden für das \textcolor{b}{Traversieren} und \textcolor{b}{Manipulieren} des Baumes.\\
	\textcolor{b}{\textbf{Window:}}\\
	Globale Variablen liegen auf dem Window. Stellt globale Objekte zur Verfügung wie \textcolor{b}{Console, History, Document etc.}\\
	\textcolor{b}{\textbf{Document:}}\\
	$\rightarrow$ Einstiegspunkt für DOM Tree. Bietet DOM-Such \& Manipulations und weitere Methoden an.\\
	\textcolor{b}{\textbf{DOM Suche/Selektion:}}
\begin{lstlisting}[style=htmlcssjs]
// Klassisch
.getElementById("idString")
.getElementsByName("nameString") //for Form-Elements
.getElementsByClassName("singleClassString") //Collection
.getElementsByTagName("tagString") // Collection
// Modern
.querySelector("selectorString") // first match
.querySelectorAll("selectorString") // NodeList
.closest("selectorString") // first matching ancestor
.matches("selectorString") // boolean
\end{lstlisting}
	Wenn mehrere Elemente zurückgegeben werden, ist dies in Form einer \textbf{HTMLCollection}:\\
	\textcolor{b}{Auswirkung:} Array Funktionen funktionieren nicht\\
	$\rightarrow$ Kein forEach, map, filter\\
	\textcolor{b}{Looping möglich:} for (elem of HTMLCollection)\{...\}\\
	$\rightarrow$ Selbes gilt für NodeList, trotzdem Möglich:
\begin{lstlisting}[style=htmlcssjs]
Array.from(navItems)
	.filter((el, i) => i % 2 === 0)
	.forEach(el => el.style.background = "green");
\end{lstlisting}
	\textcolor{b}{\textbf{DOM-Navigation:}}
\begin{lstlisting}[style=htmlcssjs]
<node>.parentElement
<node>.childNodes //auch text-Nodes und Kommentare
<node>.children //nur die Nodes vom Typ HTMLElement
<node>.firstChild //auch text-Nodes und Kommentare
<node>.firstElementChild //erstes HTMLElement
<node>.nextSibling //nächstes Geschwister-NODE
<node>.nextElementSibling //nächstes Geschwister-Element
\end{lstlisting}
\subsection{DOM-Manipulation}
	\textcolor{b}{\textbf{EventTarget:}} Können Events empfangen und/oder senden\\
	\textbf{Methoden:} addEventListener(), removeEventListener()\\
	\textcolor{b}{\textbf{Node:}} Basis Interface für finden, traversieren\\
	\textbf{Properties:} childNodes, firstChild, nextSibling\\
	\textbf{Methoden:} appendChild(), removeChild()\\
	\textcolor{b}{\textbf{Element:}} Basis für alle Elemente im Tree. (defini. Funktionen)\\
	\textbf{Properties:} id, className/classList, innerHTML, attributes\\
	\textbf{Methoden:} get/set/toggle Attribute(), closest(), queryselector, querySelectorAll, insertAdjacentHTML(), scrollTo()\\
	\textcolor{b}{\textbf{Element (ParentNode):}} Implementiert Parent \& ChildNode\\
	\textbf{Properties:} children, firstElementChild, lastElementChild\\
	\textbf{Methoden:} append(), remove()\\
	\textcolor{b}{\textbf{HTMLElement:}} Definiert HTML-Attribute \& weitere Events\\
	\textbf{Properties:} dataset, style, hidden\\
	\textcolor{b}{\textbf{Neuen Node/Element erstellen:}}
\begin{lstlisting}[style=htmlcssjs]
document.createElement("tagName") //Element mit TagName
document.createTextNode("content") //TextNode mit Inhalt
document.createDocumentFragment() //wird bei Anhängen an Parent (z.B. appendChild) entfernt. Kinder bleiben
\end{lstlisting}
	\textbf{innerHTML:} Liest oder schreibt den Inhalt vom Element als HTML Code. \textbf{innerText:} Beinhaltet nur sichtbaren Text vom Element. \textbf{textContent:} Kompletter Text vom Element.
\begin{lstlisting}[style=htmlcssjs]
// classList vs className:
<div id="el" class="box alert important"></div>
console.log(document.getElementById("el").className); // box alert important
console.log(document.getElementById("el").classList); // DOMTokenList(3) ["box", "alert", "important"]
\end{lstlisting}
\subsection{DOM-Events}
	Events ermöglichen auf Ereignisse zu reagieren. z.B. Button betätigen, Fenster verkleinert, Mauszeiger verlässt Bereich, etc.\\
	\textcolor{b}{\textbf{Events:}}
\begin{lstlisting}[style=htmlcssjs]
btn.addEventListener("keydown", (event) => console.log(event.key)); //Event prüfen (auch change)
btn.addEventListener("click", myFunc);
btn.removeEventListener("click", myFunc);
btn.addEventListener.onlick = () => alert("1");
btn.addEventListener.onclick = myFunc;
\end{lstlisting}
	\textcolor{b}{\textbf{Event Phasen:}}\\
	\textbf{1. Capture Phase:} Event durchläuft DOM Tree von Root zu Leaf, jedes Element kann reagieren.\\
	\textbf{2. Target Phase:} Das Event wird auf dem Ziel ausgelösst\\
	\textbf{3. Bubble Phase:} Das Event durchläuft den DOM Tree vom Leaf zum Root. (nicht jedes Element)\\
	\textcolor{b}{\textbf{Event-Eigenschaften:}}\\
	\textbf{target:} Zeigt auf das target-Element vom Event\\
	\textbf{currentTarget:} Zeigt auf das aktuelle (registrierte) Element\\
	\textbf{stopPropagation():} Event von Bubbling/Capturing abhalten\\
	\textbf{preventDefault():} Verhindert Default-Aktionen\\
	\textcolor{b}{\textbf{DOM Lifecycle:}}\\
	\textbf{3 Zustände:}\\
	\textbf{loading:} Dokument wird geladen.\\
	\textbf{interactive:} Dokument wurde geladen, bis auf gewisse Ressourcen. Durch Event \textcolor{b}{DOMContentLoaded} repräsentiert\\
	\textbf{complete:} Alle ressourcen wurden geladen. Event \textcolor{b}{load}
\section{Simple Single Page Apps}
	\textcolor{b}{\textbf{Form-Validation:}}
\begin{lstlisting}[style=htmlcssjs]
// Automatische Validierung abstellen mit:
<form novalidate>
function checkPwValidity(event) {
	formElement.checkValidity();//Valid. explizit anstossen
	if (!password1Element.validity.valid) //Not valid
	event.preventDefault(); ErrElm.innerText = "Error" }
	else {errElm.innerText = ""; } }
formElement.addEventListener("submit", checkPwValidity);
// Dirty JS:
const setDirty = function (event) {
	event.target.classList.add("dirty"); }
password1Element.addEventListener("blur", setDirty);
// Dirty CSS:
input.dirty:not(:focus):invalid {...}
input.dirty:not(:focus):valid {...}
\end{lstlisting}
\subsection{Konzept MVC (Model, View, Control)}
\begin{minipage}{0.48\linewidth}
	\includegraphics{mvc.png}
\end{minipage}
\begin{minipage}{0.48\linewidth}
	\textbf{Model:}\\
	Data Model (App State), Business Logic\\
	\textbf{View:}\\
	Anzeige/DOM\\
	\textbf{Controller:}\\
	User Input (Event Handler), View-Wechsel
\end{minipage}
\section{HTTP/Client-Server Interaction}
	HTTP = Hyper Text Transfer Protocol\\
	\textcolor{b}{\textbf{HTTP Request:}}\\
	Method, Address (URL Path), Protocol Version, Request Headers, Header/Body Seperator, Request Body\\
	\textcolor{b}{\textbf{GET Method:}}\\
	Um Informationen abzurufen. Daten in URL\\
	\textcolor{b}{\textbf{POST Method:}}\\
	Um Daten zu senden. Mit HTML Forms (in Request Body)\\
	\textcolor{b}{\textbf{HTTP Response:}}\\
	Protocol Version, Status Code, Response Headers, Header/Body Seperator, response Body (returning HTML Code)\\
	\textcolor{b}{\textbf{Status Codes:}} \textbf{1xx:} Informational \textbf{2xx:} Successfull (\textcolor{b}{200} OK, \textcolor{b}{201} Created, \textcolor{b}{204} No content) \textbf{3xx:} Redirection (\textcolor{b}{301} Moved Permanently) \textbf{4xx:} Client Error (\textcolor{b}{400} Bad Request, \textcolor{b}{401} Unauthorized, \textcolor{b}{403} Forbidden, \textcolor{b}{404} Not Found) \textbf{5xx:} Server Error (\textcolor{b}{500} Internal Server Error, \textcolor{b}{505} HTTP Version Not Supported) \textbf{9xx:} Non-Standard Codes\\
	\textcolor{b}{\textbf{Informationsaustausch über URL:}}\\
	http://www.example.com:80/path/to/myfile.html?\\
	key1=value1\&key2=value2\#SomewhereInTheDocument\\
	\textbf{1. Protokoll} http, https, file, mailto, tel\\
	\textbf{2. Domain Name} oder direkt ip-Adresse\\
	\textbf{3. Port} Defaults: http=80, https=443\\
	\textbf{4. Pfad} Eher logischer Pfad zu Ressource\\
	\textbf{5. Parameter} Query startet mit ?, getrennt mit \&\\
	\textbf{6. Anker} wird nicht an Server geschickt\\
	\textcolor{b}{\textbf{Node.JS:}}
\begin{lstlisting}[style=htmlcssjs]
const http = require("http");
const url = require("url");
const PORT = 8080;
function requestHandler(req, res) {
	if (req.url === "/favicon.ico") { res.end(); return } //don't server favicon.ico
	res.write("<h1>Hello ");
	res.end("World</h1>"); }
const server = http.createServer(requestHandler);
server.listen(8080, () =>
	console.log('Node listening on Port', PORT));
\end{lstlisting}
\section{AJAX}
%	\textbf{S-MVAC:} Synchronized Model View Async Controller\\
%	\includegraphics{ajax.png}
	\textcolor{b}{\textbf{Layering}}\\
	$\rightarrow$ Obere Layer nutzen untere. Die unteren nutzen nie statische Komponenten (Funktionen, Klassen) der oberen.\\
	$\rightarrow$ Rückmeldung von unteren Layern ab obere durch: \textcolor{b}{Synchron} (Return Werte von Aufrufen), \textcolor{b}{Asynchron} (Callbacks)\\
	\textcolor{b}{\textbf{Warum Layering?}}\\
	Layer ist unabhängig von oberen Layern \textbf{testbar}. Layer stellen nach Oben API zur Verfügung. Wiederverwendbarer Code.\\
	\textcolor{b}{\textbf{MVC Ergänzung:}}\\
	\textbf{updateView:} Nur diese Funktion greift auf DOM zu.\\
	\textbf{Controller:} Funktionen nutzen Variablen mit View-Referenzen, um Inhalte zu lesen und Variablen zu aktualisieren\\
	\textbf{Modell:} Status und Inhalte in Variablen. Ist passiv und synchronisiert (AJAX)\\
	\textcolor{b}{\textbf{Herausforderungen bei AJAX Single Page Apps:}}\\
	\textcolor{b}{- Layering:} ok\\
	\textcolor{b}{- Langsamer API Service:}\\
	\textbf{Abhilfe:} Nach Auslösen der Anfrage: View geht in \textcolor{b}{Suspense Status}, \textcolor{b}{Disable} der entsprechenden UI-Controls, \textcolor{b}{Feedback}, dass die Anfrage läuft, allfällig vorherige Werte entfernen\\
	\textcolor{b}{- Unzuverlässiger API Service:}\\
	\textbf{Abhilfe:} Bei Fehlerhaften Antworten (not «ok»)	\textcolor{b}{Request wiederholen} (retry). \textcolor{b}{Zeitlimit} (Timeout) setzen und nachher Anfrage abbrechen\\
	\textcolor{b}{- Langsamer API Servi bei kontinuierlicher Nutzerinteraktion:}\\
	\textbf{Abhilfe:} \textcolor{b}{Debouncing/Throttling} $\rightarrow$ nach jedem Tastendruck kurz (200ms) warten. Request \textcolor{b}{Ersetzen} (aktuellen abbrechen)\\
	\textcolor{b}{- Server-Updates an Clients verteilen (Server Push):}\\
	\textbf{Abhilfe:} \textcolor{b}{Polling:} Im Hintergrund wird wiederholt der Server für neue Information angefragt. \textcolor{b}{WebSockets:} Es wird eine zweiseitige Verbindung mit dem Server aufgebaut.\\
	\textcolor{b}{- HTML Injection/Cross Site Scripting (XSS):}\\
	\textbf{Abhilfe:} \textcolor{b}{Server:} Säuberung von erhaltenen Inhalten, Conten Security Policy (CSP). \textcolor{b}{Säuberung im Client:} Bei Einfügen von externen Inhalten.
\subsection{AJAX Grundlagen}
\begin{lstlisting}[style=htmlcssjs]
// Promise Grundaufbau:
const promise = new Promise((resolve, reject) => {
	if(...) {resolve("Seccess")} else {reject("Error")} });
promise.then((value) => {console.log("ok" + value);})
	.catch((value) => {console.log("Error" + value); });
// Callback Functions: (Funktion mit Funktion als Param)
let x = function () {console.log("Callback"); }
let y = function (TestCallbackFn) {
	console.log("first"); TestCallbackFn(); }
y(x);
// Callback mit Promise:
function promisedWait() {
	return new Promise((resolve, reject) => {x() => {resolve(), reject()} } )}
promisedWait.then(..).catch(..);
\end{lstlisting}
	\textcolor{b}{\textbf{Promises:}}\\
	Ist ein Objekt mit zwei Callback-Funktionen: resolve, reject
	\textcolor{b}{\textbf{Async/Await:}}\\
	Async bezeichnet eine Funktion, die im Hintergrund weiterlaufen kann. $\rightarrow$ Verwendet, um Fetch (return Promise) abzuwarten (mittels await)\\
	\textcolor{b}{\textbf{Fetch:}}\\
	Ermöglicht das Ansprechen von Web-Resourcen im Browser mittels JS. Asynchron (wie setTimeout), aber statt Callback-Funktion als Argument. Return eines Promise
\begin{lstlisting}[style=htmlcssjs]
// Einfacher Request (then, catch)
fetch(API_URL).then((response) => {
	if (response.ok) {return response.json();}
	else {return Promise.reject();} })
.then((data) => {console.log('thenData', data); })
.catch((error) => console.log('thenError'))
// Einfacher Request (await, try/catch)
async function test () {try {
	const response = await fetch(API_URL);
	const data = await response.json();
	console.log('awaitData', data);
}catch{console.log('thenError'); }}
//Abortable Fetch
const controller = new AbortController();
const signal = controller.signal;
setTimeout(() => controller.abort(), 5000);
fetch(url, {signal}).then(response => {
	return response.text();
}).then(text => {console.log(text); });
\end{lstlisting}
	\textcolor{b}{\textbf{Revealing Module Pattern:}}\\
	Nur die extern wichtigen Variablen und Funktionen exportieren
\begin{lstlisting}[style=htmlcssjs]
(function () {
let hiddenCounter = 0;
function getCounter () {return hiddenCounter;}
function incCounter () {return ++hiddenCounter;}
// export
window.counterService = {getCounter, incCounter}; })()
\end{lstlisting}
	\textcolor{b}{\textbf{ES6 Modules:}}\\
	Moderne Variante für Modularisierung in JavaScript.
\begin{lstlisting}[style=htmlcssjs]
var hiddenVarVar = "VarVar";
let hiddenCounter = 0;
export function getCounter() {return hiddenCounter;}
export function incCounter() {return ++hiddenCounter;}
//In other File:
<script src="ServiceES6.js" type="module"></script>
<script type="module">
	import {getCounter, incCounter} from './ServiceES6.js';
</script>
\end{lstlisting}
	\textcolor{b}{\textbf{Fetch Alternativen:}}\\
	XMLHttpRequest API, JQuery, Axios\\
	\textcolor{b}{\textbf{Same Origin Policy (SOP):}}\\
	Server können in ihrer Antwort an den Browser anzeigen ob sie eine Anfrage von dieser "Origin" (URL der Seite) zulassen wollen.\\
	Wenn der Server in der Antwort keine Information zur Access-Control-Allow-Origin mitschickt	gilt die Same-Origin-Policy.	D.h. es dürfen sich nur Seiten verbinden, die auch von diesem Server geservt wurden.\\
	\textcolor{b}{\textbf{Cross Origin Resource Sharing (CORS):}}\\
	Wenn ein Server einer anderen Origin kontaktiert wird (nicht SOP), muss dieser im Header der Antwort das Feld Access-Control-Allow-Origin auf die Origin des Browsers	gesetzt haben (oder auf *).\\
	$\rightarrow$ Nur dann darf der Browser die	Antwort entgegennehmen.
\section{REST (Representational State Transfer)}
	REST is a software architecture style consisting of guidelines and best practices for creating scalable web	services.\\
	\textbf{Vor REST}: Viele verschiedene Standards. Schlechte Interoperabilität. Rad wurde immer wieder neu erfunden. Vendor lock-in\\
	\textcolor{b}{\textbf{Eigenschaften:}}\\
	\textbf{Client/Server:} Trennung von Client- und Server-Logik\\
	\textbf{Statuslose Kommunikation:} Jeder request beinhaltet alle möglichen Informationen\\
	\textbf{Cache'bar:} Clients können antworten zwischenspeichern. Sofern dies erlaubt wurde\\
	\textbf{Layered System:} Erlaubt Einsatz von Schichtenarchitekturen inkl. Load-Balancers, Proxies und Firewalls\\
	\textbf{Uniform Interface:} Einheitliche Schnittstelle zw. Client \& Server (Selbstbeschreibend)\\
	\textcolor{b}{ROA: Standardmethoden}\\
	\textbf{GET:} Ressource wird angefordert\\
	\textbf{POST:} Erzeugt neue Ressource\\
	\textbf{PUT:} Aktualisiert/erzeugt eine Ressource\\
	\textbf{DELETE:} Löscht eine Ressource\\
	\textbf{OPTIONS:} Gibt an, wie die Ressource verwendet werden darf (HTTP-Methoden)\\
	\textbf{PATCH:} Partielles updaten einer Ressource\\
	\textbf{HEAD:} Gleich wie GET, jedoch ohne Ressource zu erhalten (Für Caching)\\
	\textcolor{b}{\textbf{HATEOAS:}}\\
	$\rightarrow$ Hypermedia As The Engine OF Application State
	\begin{itemize}[topsep=0pt, leftmargin=3mm]
		\setlength\itemsep{-0.3em}
		\item Prozessgedanke in der Ressource
		\item Media-Typen beschreiben die Ressource
		\item Aktionen werden ausgeführt beim folgen von Links
		\item Jede Antwort beinhaltet den "Application State"
		\item Selbstbeschreibende API’s erzeugen Flexibilität
		\item Clients können die API "erforschen" ohne Dokumentation und Anleitung
	\end{itemize}
	\textbf{Vorteile:}Inline Dokumentation, Exportable API, Einfachere Clients, die URI ist sicher korrekt und aktuell, URI kann, falls gewollt, Serverseitig einfach geändert werden.\\
	\textbf{Nachteil:} (Sehr) aufwändig auf der Server-Seite
\section{USER EXPERIENCE (UX)}
	\textcolor{b}{\textbf{Design Probleme:}}
	\begin{itemize}[topsep=0pt, leftmargin=3mm]
		\setlength\itemsep{-0.3em}
		\item Schlechte Führung des Auges $\rightarrow$ Optionen schwierig zu unterscheiden, Fehlende Grupierung/visuelle Unterscheidung
		\item Schlechte $"$Affordances$"$ (Greifbarkeit)
		\item Fehlende Constraints
		\item Fehlendes Undo
	\end{itemize}
	\textcolor{b}{\textbf{Techniken zur Aufgenführung:}}\\
	Standardreihenfolge: Oben $\rightarrow$ Unten, Links $\rightarrow$ Rechts\\
	Eye Catcher helfen die Standardreihenfolge zu verändern:
	\begin{itemize}[topsep=0pt, leftmargin=3mm]
		\setlength\itemsep{-0.3em}
		\item helle Elemente $\rightarrow$ dunkle
		\item Einzelnes $\rightarrow$ Gruppe
		\item Grafik $\rightarrow$ Text
		\item Farbe $\rightarrow$ Schwarz-Weiss
		\item satte Farben $\rightarrow$ unsatte
		\item dunkle Flächen $\rightarrow$ helle
		\item gross $\rightarrow$ klein
		\item ungewohnt $\rightarrow$ gewohnt
		\item Bewegung
		\item Augen \& Blickrichtung
		\item Gesichter \& Mimik
	\end{itemize}
	\textcolor{b}{\textbf{Gutes Interaktionsdesign:}}\\
	zeigt den aktuellen \textbf{Zustand} des Systems, zeigt dem Nutzer \textbf{Möglichkeiten} der Interaktion, spricht die \textbf{Sprache} der Nutzer, begleitet den Nutzer \textbf{Schritt für Schritt} zum Ziel.\\
	\textcolor{b}{\textbf{Interaktionselemente Controls:}}\\
	Sichtbar, wahrgenommen, zielführend, genutzt\\
	\textcolor{b}{\textbf{Touch Targets:}}\\
	Should be at least 1cm$^{2}$. Best: 0.9cm + 0.2cm padding\\
	$\rightarrow$ More space needed when used in stressful situations\\
	\textcolor{b}{\textbf{Benutzerführung:}}\\
	Constraints, Confirmation, Undo\\
	Professionelle Entwickler erwarten vom (bzw. entwickeln mit dem) $"$UX Team$"$ (+ Product Management):
	\begin{itemize}[topsep=0pt, leftmargin=3mm]
		\setlength\itemsep{-0.3em}
		\item Styleguide/Design System
		\item Site/App Design
		\item Nutzungs-Szenarien mit Benutzerprofilen ($"$Personas$"$)
		\item Usability Tests (aussagekräftig, früh genug, gut kommuniziert)
	\end{itemize}
	\textcolor{b}{\textbf{Visuelles Design:}}\\
	Farbblindheit, Fehlender Kontrast, gewisse Farbkombinationen (z.B. \textcolor{red}{BLAU} auf \textcolor{b}{ROT})
\section{Handlebars}
\begin{lstlisting}[style=htmlcssjs]
<script id="template" type="text/x-handlebars-template">
	<ul> {{#each this}} <li> <h3> {{rating}}
	<button data-delta="1" data-song-id="{{id}}">+</button>
	<button data-delta="-1"data-song-id="{{id}}">-</button>
	{{title}} </h3> <p>{{artist}}</p> </li> {{/each}} </ul>
</script>
// Template erstellen
const templateSource = document.getElementById("template").innerHTML;
const template = Handlebars.compile(templateSource);
//Template-Function kann aufgerufen werden:
container.innerHTML = template(anyObject or function());
\end{lstlisting}
\section{Countdown mit Callback}
\begin{lstlisting}[style=htmlcssjs]
function StartCountdown(startValue, fnTick, fnFinish) {
	let currentValue = startValue;
	function countDown() {
		if (currentValue > 0) {
			setTimeout(() => {
				fnTick(--currentValue);
				countDown(currentValue)
			}, 1000)
		} else {fnFinish(startValue) } }
	countDown(); }
\end{lstlisting}
\includegraphics[width=0.7\linewidth]{spezifitaet.png}
\end{multicols*}
\end{document}
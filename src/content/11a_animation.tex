%! Author = joels
%! Date = 03/01/2021

\section{Animation}
% Studierende können …
% - Animationen mit CSS-Transition-Properties (transition-* und transition) definieren, sowie Fehler in vorgegebenen CSS Regel mit Transition-Properties erklären und korrigieren.
% - die Unterschiede der wichtigsten Werte des Properties transition-timing-function (linear, easein, ease-out, ease-in-out) erklären und einer entsprechenden Visualisierung des Animationsablaufes zuordnen.
% - vorhersagen wie sich CSS transform definitionen (rotate(), rotateX(), rotateY(), translate(), translateX(), translateY(), scale(), scaleX(), scaleY(), skew(), skewX(), skewY(), none) in Kombination mit unterschiedlichen transform-origin Werten (z.B. left, center, right, top) auf die Darstellung eines HTML Elementes auswirken.
% -vorhersagen wie sich eine einfache, mittels einer @keyframe Regel beschriebenen Animation abläuft.
% - die vermittelte Heuristik anwenden um zu bestimmen ob ein CSS-Property animierbar ist.
% - erklären warum es nötig sein kann ein mit einer @property Definition einen Wertebereich für CSSCustom Property zu definieren.
% - erklären warum die Animation des opacity-Properties gut geeignet ist für Fade-In und Fade-Out Animationen, aber zusätzliche Vorkehrungen getroffen werden müssen um die Accessibility sicher zu stellen.
% - erklären warum CSS-Animationen JS-basierten Animationen vorzuziehen sind.

\textbf{\textcolor{blue}{Transition Properties:}}
\begin{itemize}[topsep=0pt, leftmargin=3mm]
    \setlength\itemsep{-0.3em}
    \item \textbf{tranistion-property:} Welches CSS property geändert wird (z.B. background-color, all)
    \item \textbf{tranistion-duration:} Dauer in s oder ms
    \item \textbf{tranistion-timing-function:} Verhalten (ease, linear, ease-in, ease-out, ease-in-out, step-start, step-end, steps( ... ), cubic-bezier(\#,\#,\#,\#))
    \item \textbf{tranistion-delay:} Delay in s oder ms
    \item \textbf{transition: property duration timing-function delay}
    \item \textbf{Mehrere Transition Properties:} Mit Komma getrennt
    \item \textbf{transform:} Ändert die Form (rotate[X|Y](), translate[X|Y](), scale[X|Y](), skew[X|Y](), none)
\end{itemize}
\textbf{\textcolor{blue}{Keyframe Animation:}}\\
Ablauf einer Animation kann definiert werden:\\
$\rightarrow$ Nicht animierbar: border-style, display (weitere Elemente, die \dq value\dq benutzen)
\begin{lstlisting}[style=htmlcssjs]
@keyframes rainbow { // name: rainbow
    0% { background-color: red; }
    20% { background-color: orange; }
    40% { background-color: yellow; }
    60% { background-color: green; }
    80% { background-color: blue; }
    100% { background-color: purple;} }
#magic { // keyframe benutzen:
    animation-name: rainbow;
    animation-duration: 5s;
    animation-timing-function: linear;
    animation-iteration-count: infinite;
    animation-direction: alternate; }
\end{lstlisting}

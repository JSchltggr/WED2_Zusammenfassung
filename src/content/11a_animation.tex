%! Author = joels
%! Date = 03/01/2021

\section{Animation}
% Studierende können …
% - Animationen mit CSS-Transition-Properties (transition-* und transition) definieren, sowie Fehler in vorgegebenen CSS Regel mit Transition-Properties erklären und korrigieren.
% - die Unterschiede der wichtigsten Werte des Properties transition-timing-function (linear, easein, ease-out, ease-in-out) erklären und einer entsprechenden Visualisierung des Animationsablaufes zuordnen.
% - vorhersagen wie sich CSS transform definitionen (rotate(), rotateX(), rotateY(), translate(), translateX(), translateY(), scale(), scaleX(), scaleY(), skew(), skewX(), skewY(), none) in Kombination mit unterschiedlichen transform-origin Werten (z.B. left, center, right, top) auf die Darstellung eines HTML Elementes auswirken.
% -vorhersagen wie sich eine einfache, mittels einer @keyframe Regel beschriebenen Animation abläuft.
% - die vermittelte Heuristik anwenden um zu bestimmen ob ein CSS-Property animierbar ist.
% - erklären warum es nötig sein kann ein mit einer @property Definition einen Wertebereich für CSSCustom Property zu definieren.
% - erklären warum die Animation des opacity-Properties gut geeignet ist für Fade-In und Fade-Out Animationen, aber zusätzliche Vorkehrungen getroffen werden müssen um die Accessibility sicher zu stellen.
% - erklären warum CSS-Animationen JS-basierten Animationen vorzuziehen sind.
%! Author = joels
%! Date = 03/01/2021

\section{Security}
% Studierende können …
% - die folgenden wichtige Verletzlichkeiten von server-seitigen Web-Anwendungen erklären und im Code einfacher Express-Anwendungen identifizieren und beheben
    % - Stored XSS
    % - Code-Injection / Remote Code Execution
    % - Broken Autorisation:
        % - Cross-Site Request Forgery
    % - Broken Access Control
        %- Insecure Direct Object References
        %- Replay Attacken
% - beurteilen und begründen ob es Sinn macht für eine Anwendung einen passwortbasierten «Authentication Service» selber zu betreiben.
% - wichtige generelle Sicherheitsmassnahmen für Node/Express Anwendungen aufzählen, erklären und deren Fehlen im Code einfacher Express-Anwendungen identifizieren und beheben.
\textbf{\textcolor{blue}{XSS (Cross-Site Scripting):}}\\
\textbf{Definition:} Den Server so manipulieren, dass Schadcode (JavaScript) an Nutzer (Opfer) ausgeliefert wird und im Browser dieser Nutzer ausgeführt wird.\\
\textbf{Gegenmassnehmen:}
\begin{itemize}[topsep=0pt, leftmargin=3mm]
    \setlength\itemsep{-0.3em}
    \item XSS oder DOMPurify Library
    \item \dq Encoding\dq bei der Darstellung (Output) von Nutzer-Input
    \item Content Security Policy (CSP) im Header setzen
    \item Bei Cookies soweit möglich das HTTPOnly Flag setzen
\end{itemize}
\textbf{\textcolor{blue}{Code-Injection/Remote Code Execution:}}\\
\textbf{Definition:} Den Server dazu bringen, dass eingeschleuster Code ausgeführt wird.\\
\textbf{Gegenmassnehmen:}
\begin{itemize}[topsep=0pt, leftmargin=3mm]
    \setlength\itemsep{-0.3em}
    \item NICHT \textcolor{blue}{eval()}, sondern \textcolor{blue}{parseInt(), JSON.parse()} nutzen
    \item Globale Scopes und Variablen reduzieren.
    \item Rechenintensive Tasks mit childprocess.spawn auslagern, um (D)DOS Attacken zu erschweren
    \item Node NICHT als root Prozess starten
\end{itemize}
\textbf{\textcolor{blue}{Stored Broken Authentication:}}\\
\textbf{Ziel:} Bereiche eines Web-Sites mit benutzer-spezifischen Informationen sollten nur für authentisierte Nutzer zugreifbar sein.\\
\textbf{Gegenmassnehmen:}
\begin{itemize}[topsep=0pt, leftmargin=3mm]
    \setlength\itemsep{-0.3em}
    \item Keine geheimen Informationen in query-parametern
    \item https (TLS) nutzen
    \item Authentication-Service nutzen
    \item Session Timeout sinnvoll setzen
    \item Formulare beim Ausliefern mit einem Token versehen
\end{itemize}
\textbf{\textcolor{blue}{Stored Broken Access Control:}}
\textbf{Ziel:} Bereiche eines Web-Sites mit benutzer-spezifischen Informationen sollten nur für autorisierte Nutzer zugreifbar sein (read)\\
\textbf{Gegenmassnehmen:}
\begin{itemize}[topsep=0pt, leftmargin=3mm]
    \setlength\itemsep{-0.3em}
    \item Sicher stellen,dass der eingeloggte Nutzer berechtigt ist.
    \item Token mit einmaliger Gültigkeit
\end{itemize}
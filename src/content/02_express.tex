%! Author = joels
%! Date = 03/01/2021

\section{Week01}
% Studierende können …
%• … sich in einer mit Express CLI erstellten Anwendung zurechtfinden, diese komplettieren sowie
%Fehler identifizieren und beheben.
%• … das Konzept von Express-Middlewares erklären, Middlewares einbinden, Fehler in der Einbindung von
%Middlewares diagnostizieren und beheben (z.B. Reihenfolge der Registrierung, Nutzung von next(),
%Unterschied Error-Middleware und “normale” Middleware).
%• … die folgenden Express Middlewares einsetzen: body-parser, cookie-parser, express-session,, errorhandler,
%csurf, express.static, express.Router .
%• … das von Express bereitgestellt Request und Response API nutzen.
%• … können den Unterschied zwischen Express Request Parametern und URL Query Parametern und
%deren Einsatzbereich erklären sowie beide Typen von Parametern in einem Request-Handler auslesen.
%• … eine NEDB Datenbank (und ähnliche Backend-File-Zugriffe) in ein Backend-Service-Modul kapseln.
%• … den Einsatzbereich, Unterschiede und Ähnlichkeiten von http-Cookies und dem Express-Session
%Objekt erklären und im Code z.B. die Nutzung von Cookies durch die Nutzung von des Session Objekts
%ersetzen.
%• … einen einfachen REST Service mit Express erstellen und diese REST-Schnittstelle mit einem AJAXClient
%(mittels fetch) ansprechen.
%• … erklären in welchen Situationen sich die Nutzung Web-Sockets anbietet.
%• ... die res.render Methode in Kombination mit der HandleBars Template-Engine einsetzen
%(Datenübergabe über das Context-Objekt im Aufruf der render-Funktion sowie über res.locals)
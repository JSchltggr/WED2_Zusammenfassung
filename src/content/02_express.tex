%! Author = joels
%! Date = 03/01/2021

\section{Express}
% Studierende können …
%• … sich in einer mit Express CLI erstellten Anwendung zurechtfinden, diese komplettieren sowie
%Fehler identifizieren und beheben.
%• … das Konzept von Express-Middlewares erklären, Middlewares einbinden, Fehler in der Einbindung von
%Middlewares diagnostizieren und beheben (z.B. Reihenfolge der Registrierung, Nutzung von next(),
%Unterschied Error-Middleware und “normale” Middleware).
%• … die folgenden Express Middlewares einsetzen: body-parser, cookie-parser, express-session,, errorhandler,
%csurf, express.static, express.Router .
%• … das von Express bereitgestellt Request und Response API nutzen.
%• … können den Unterschied zwischen Express Request Parametern und URL Query Parametern und
%deren Einsatzbereich erklären sowie beide Typen von Parametern in einem Request-Handler auslesen.
%• … eine NEDB Datenbank (und ähnliche Backend-File-Zugriffe) in ein Backend-Service-Modul kapseln.
%• … den Einsatzbereich, Unterschiede und Ähnlichkeiten von http-Cookies und dem Express-Session
%Objekt erklären und im Code z.B. die Nutzung von Cookies durch die Nutzung von des Session Objekts
%ersetzen.
%• … einen einfachen REST Service mit Express erstellen und diese REST-Schnittstelle mit einem AJAXClient
%(mittels fetch) ansprechen.
%• … erklären in welchen Situationen sich die Nutzung Web-Sockets anbietet.
%• ... die res.render Methode in Kombination mit der HandleBars Template-Engine einsetzen
%(Datenübergabe über das Context-Objekt im Aufruf der render-Funktion sowie über res.locals)
\textbf{\textcolor{blue}{Server starten:}}
\begin{lstlisting}[style=htmlcssjs]
import http from "http";
import express from "express";
const app = express();
const server = http.createServer(app);
const hostname = '127.0.0.1'; const port = '3000';
server.listen(port, hostname, () => {
  console.log('Running at http://${hostname}:${port}/');
});
\end{lstlisting}
\textbf{\textcolor{blue}{JSON (JavaScript Object Notation):}}
\begin{itemize}[topsep=0pt, leftmargin=3mm]
    \setlength\itemsep{-0.3em}
    \item Ist ein Daten-Austauschformat
    \item Wird verwendet um Daten zu senden und speichern
    \item Hat im Web XML verdrängt
    \item Wird oft mit AJAX verwendet
    \item Datentypen: String, Number, Boolean, Array, Object, null
    \item JSON-Helper: JSON.parse \& JSON.stringify
\end{itemize}
\textbf{\textcolor{blue}{MVC-Pattern:}}
\begin{itemize}[topsep=0pt, leftmargin=3mm]
    \setlength\itemsep{-0.3em}
    \item Model: Daten und Datenaufbereitung
    \item Controller: Verknüpft die View mit den Daten
    \item View: Darstellen der daten
\end{itemize}
\textbf{\textcolor{blue}{Middleware:}}
\begin{itemize}[topsep=0pt, leftmargin=3mm]
    \setlength\itemsep{-0.3em}
    \item Wird für Request Bearbeitung gebraucht
    \item Middleware ist ein Stack von Anweisungen welche für einen Request ausgeführt wird
    \item Neue Middleware registrieren mit: app.use(..);
\end{itemize}
\subsection{Routing}
\textbf{\textcolor{blue}{Router-Middleware:}}
\begin{lstlisting}[style=htmlcssjs]
// Middleware befindet sich auf dem Express Objekt
import express from "express";
const router = express.Router;
// HTTP Methoden (get, put, post, delete)
router.get('/', function(req, res) {
    res.send('hello world'); });
// Mehrere Methoden auf selbem Link mit .route
app.route('/book') // oder router.route(..)
    .get(function(req, res) {res.send('Get a book');})
    .post(function(req, res) {res.send('Add a book');})
\end{lstlisting}
\textbf{\textcolor{blue}{BodyParser-Middleware:}}
\begin{lstlisting}[style=htmlcssjs]
import bodyParser from "body-parser";
app.use(bodyParser.json());
\end{lstlisting}
\textbf{\textcolor{blue}{Static-Middleware:}}
\begin{itemize}[topsep=0pt, leftmargin=3mm]
    \setlength\itemsep{-0.3em}
    \item Aufgabe: Statische Files ausliefern
    \item Nutzen wie folgt:\\
    app.use(express.static($\_$$\_$dirname + '/public'));\\
    app.use(express.static(path.join(path.resolve(), 'public')));
    \item Es sind mehrere static-routes möglich
\end{itemize}
\textbf{\textcolor{blue}{Custom-Middleware:}}\\
Hat 3 Parameter: request, response, next
\begin{lstlisting}[style=htmlcssjs]
function myDummyLogger( options ){
    options = options ? options : {};
    return function myInnerDummyLogger(req, res, next) {
        console.log(req.method +":"+ req.url);
        next(); } }
app.use(myDummyLogger());
\end{lstlisting}
\textbf{\textcolor{blue}{Error-Middleware:}}
\begin{itemize}[topsep=0pt, leftmargin=3mm]
    \setlength\itemsep{-0.3em}
    \item Bearbeitet Errors, welche von Middlewares generiert werden
    \item Hat 4 Parameter: error, request, response \& next
    \item Sollte als letzte Middleware registriert werden
    \item Wird aufgerufen, falls ein error-Objekt dem Next-Callback übergeben wird.
\end{itemize}
\begin{lstlisting}[style=htmlcssjs]
app.use(function(err, req, res, next) {
    console.error(err.stack);
    res.status(500).send('Something broke!'); });
\end{lstlisting}
\subsection{Model}
\textbf{Ziel:} Die Daten sollten in einem Module verwaltet und abgespeichert werden. Möglichkeiten: In Memory (array), JSON, NoSQL-Datenbanken (nedb), Sql-Datenbanken, Oracle-datenbank.
\begin{lstlisting}[style=htmlcssjs]
// beispiel: nedb
import Datastore from "nedb";
const db = new Datastore({ filename: './data/order.db', autoload: true });
// insert
db.insert(order, function(err, newDoc) {
    if(callback) {
        callback(err, newDoc); } });
// search: findOne oder findAll
db.findOne({ _id: id }, function (err, doc) {
    callback( err, doc); });
// update
db.update({_id: id}, {$set: {"state": "DELETED"}}, {}, function (err, doc) {
    publicGet(id,callback); });
\end{lstlisting}
\subsection{View}
\textbf{Ziel:} Trennen von Controller und View mittels \textcolor{blue}{Template Engine}. Express bietet eine render Methode an: app.render(view, [locals], callback);
\begin{lstlisting}[style=htmlcssjs]
// view engine setup
app.set('views', path.join(path.resolve(), 'views'));
app.set('view engine', 'hbs');
\end{lstlisting}
\subsection{Session \& Security}
\textbf{\textcolor{blue}{Session:}}\\
Beim ersten \dq Connect\dq vom Client wird eine Session-Id erstellt und als Cookie zum Client geschickt. Die Session-Daten werden auf dem Server abgespeichert. $\rightarrow$ Wiederspricht REST\\
\textbf{Nutzen:} HTTP-Stateless umgehen und z.B. Login Status von user abspeichern, oder allgemein Daten Server-Seitig einem Benutzer zuordnen. Ermöglicht tracking.
\begin{lstlisting}[style=htmlcssjs]
// Cookie verwenden:
app.use(require("cookie-parser")());
// Session benötigt Cookies
app.use(session({ secret: ‚'1234567', resave: false, saveUninitialized: true}));
\end{lstlisting}
\subsection{Rest \& Ajax}
\textbf{\textcolor{blue}{Token:}}\\
\textbf{Ziel:} Stateless Server. \textbf{Idee: Bei jeder Anfrage muss für die Authorisierung ein Token mitgegeben werden.} \textbf{Vorteil:} Jede Anfrage kann zu einem beliebigem Server gesendet werden. \textbf{Nachteile:} Was passiert wenn der Token geklaut wird? $\rightarrow$ Ablaufdatum kurz setzen, Token invalidieren.\\
\textbf{\textcolor{blue}{JWT-Token:}} import jwt from 'express-jwt';
\begin{lstlisting}[style=htmlcssjs]
// fetch
fetch('/login', {
    method: 'POST',
    headers: { 'Content-Type': 'application/json' },
    body: JSON.stringify({email: "a@b.ch", pwd: "123"})
}).then(function (res) { console.log(res); });
// Cookies werden nicht automatisch mitgeschickt
fetch('https://example.com', {
    credentials: 'include'
});
\end{lstlisting}
\subsection{Web Sockets}
Das klassische Model vom Request-Response hat 2 Probleme: Der Server kann keine Nachricht an den Client schicken. Jede Anfrage öffnet eine neue Verbindung.\\
Dieses Model erschwert es real-time Apps zu machen (Games, Chats). Lösung: WebSockets ermöglichen \dq bi-directional\dq, \dq always-on\dq Kommunikation.
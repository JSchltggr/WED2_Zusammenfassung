%! Author = joels
%! Date = 03/01/2021

\section{Accessibility}

% Studierende können …
% - Techniken aufzählen wie Web-UI sinnvoll für Farbenblindeheit optimiert werden können.
% - erklären warum Optimierung für Farbblindheit und Optimierung des Farbkontrasts von Web-UI Elementen nicht das Gleiche ist .
% - Überprüfungsvorschläge des Chrome Accessibility Checks in einem Kontext in Handlungsempfehlungen übersetzen.
% - erklären warum Zoombarkeit von Web-Uis wichtig ist, und welche Praktiken vermieden werden sollten.
% - erklären warum Animationen in Web-UI abgestellt werden können sollen.
% - die Accessibility von Formularen und Tabellen beurteilen und optimieren.

\textbf{Für Elemente zu beachten:} Nicht-Text-Inhalte mit Alt-Tag versehen, Tastaturbedienbarkeit, Logische Reihenfolge, Semantische Struktur, Flexibilität der Anzeige, Kontrast, Verständlichkeit, Konsistenz/Vorhersehbarkeit, Syntax/Kompatibilität, Hilfestellung bei Interaktionen, PDF Accessibility
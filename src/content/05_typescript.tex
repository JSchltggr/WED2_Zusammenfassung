%! Author = joels
%! Date = 03/01/2021

\section{Typescript}

% Studierende können …
% - können die Vorteile der Nutzung von TypeScript in einem gegebenen Beispielkontext erklären
% - können Typescript mit TSLint und Prettier für ihr Testat nutzen
% - können TypeScript Code auf der Komplexitätsstufe der Übungen lesen und Vorhersagen zur Identifikation von Typenfehlern durch TypeScript machen
% - TypeScript Klassendefinitionen lesen und in entsprechende Klassendefinitionen in EcmaScript-2015+ (ES6+) konvertieren.
% - können den Unterschied von Klassen und Interfaces (und Typen) und das Konzept des “Structural Typing“ in TypeScript erklären.
% - können das Konzept von Union-Types erklären und in Beispielen nutzen.

\textbf{\textcolor{blue}{Variablendeklarationen mit Basistypen:}}
\begin{itemize}[topsep=0pt, leftmargin=3mm]
    \setlength\itemsep{-0.3em}
    \item Variablen können Typdeklaration erhalten. Ohne Typdeklaration wird die \dq Type-Inferenz\dq verwendet.
    \item Variablen können explizit als \textcolor{blue}{any} deklariert werden. $\rightarrow$ Beliebige Werte dürfen zugewiesen werden. Entsprechend geht auch die Zuweisung an jede andere Variable.
    \item Globale Variablen aus nicht TS-Files können mit dem Keyword \textcolor{blue}{declare} deklariert werden.
    \item Basis-Typen: \textbf{boolean, number, string, null, undefined}
\end{itemize}
\begin{lstlisting}[style=htmlcssjs]
// Beispiele
let myAnyVar1: any = 1;
let myInferredNumVar1 = 1;
let myNumberVar: number = 1;
let myStringVar: string = 'cdef';
declare let myMagicVar: string;
// allowed
myAnyVar1 = 'hi';
myNumberVar = myAnyVar1; // might come as surprise
// not allowed
myInferedNumVar1 = 'hi';
myStringVar = myInferedNumVar1;
myStringVar = 1;
myNumberVar = 'hi';
\end{lstlisting}
\textbf{\textcolor{blue}{Variablendeklarationen mit komplexen Typen:}}
\begin{itemize}[topsep=0pt, leftmargin=3mm]
    \setlength\itemsep{-0.3em}
    \item Typescript erlaubt die Deklaration von \textbf{Arrays, Tupels und Enums}
    \item Bei Tupeln wird keine Type-Inferenz angewendet
    \item Enums können wie Basistypen genutzt werden
    \item Default-Repräsentation: Integers (Strings möglich)
    \item Alternative: String Literal Type
\end{itemize}
\begin{lstlisting}[style=htmlcssjs]
// Beispiele
let myInferredNumArray = [1, 2, 3];
let myNotInferredTupel = [1, 'abcd'];
let myNumArray: number[] = [1, 2, 3];
let myTupel: [number, string] = [1, 'abcd']
enum Color {Red, Green, Blue};
enum StrColor {Red = "red", Green = "green"};
type StrLitColor = "red" | "green";
let c: Color = Color.Green;
let myTupel2: [Color, number] = [Color.Green, 1];
// allowed
myNotInferredTupel[0] = 2;
myNotInferredTupel[1] = 2; // not inferred
myTupel[1]='hi';
//not allowed
myInferredNumArray[2] = 'hi';
myInferredNumArray[4] = 'hi';
myTupel[0]= 'hi';
myTupel[1]= 2;
\end{lstlisting}
\textbf{\textcolor{blue}{Funktionsdeklarationen:}}
\begin{itemize}[topsep=0pt, leftmargin=3mm]
    \setlength\itemsep{-0.3em}
    \item Spezifizierbar: Typen der Parameter + Typ des Rückgabewertes
    \item Mehr als eine erlaubte Signatur pro Funktion möglich !
    \item Funktionsparameter können optional sein (? direkt nach dem Namen der Variablen )
\end{itemize}
\begin{lstlisting}[style=htmlcssjs]
// Beispiele
function add(s1: string, s2: string): string;
function add(n1: number, n2: number): number;
function add(n1, n2) {
    return n1 + n2; }
function combineFunction(sn: number | string = "", ns?: number): string {
    return String(sn) + String(ns || ""); }
// allowed
let myNum: number = add(1, 2);
combineFunction(1);
combineFunction('hi', 3);
// not allowed
let myStr: string = add(1, 2);
let myNum: number = add("kk", 2);
combineFunction(1, 'hi');
\end{lstlisting}
\textbf{\textcolor{blue}{Funktionen als Parameter:}}
\begin{itemize}[topsep=0pt, leftmargin=3mm]
    \setlength\itemsep{-0.3em}
    \item Funktionsparameter können Funktionen sein
    \item Signatur dieser Parameter kann auch deklariert werden
\end{itemize}
\begin{lstlisting}[style=htmlcssjs]
// Beispiele
function numberApplicator(numArray: number[],numFun: (prevRes: number, current: number) =>number): number{
    return numArray.reduce(numFun); }
function concatFunction(s1: string, s2: string): string {
    return s1 + s2; }
// allowed
let myNum2: number = numberApplicator([1, 2, 3, 4], add);
// not allowed
numberApplicator([1, 2, 3, 4], concatFunction);
\end{lstlisting}
\textbf{\textcolor{blue}{Klassen:}}
\begin{itemize}[topsep=0pt, leftmargin=3mm]
    \setlength\itemsep{-0.3em}
    \item Properties der Instanzen und der Klasse (Static) werden im Kontext der Klasse definiert.
    \item Methoden und Properties können mit den Zusätzen "private" und "readonly" versehen werden.
\end{itemize}
\begin{lstlisting}[style=htmlcssjs]
class Counter {
  private _doors: number;
  public static readonly WOOD_FACTORS = {'oak': 80, 'pine': 20};
// public static readonly MIN_DOOR_COUNT = // code
  constructor({doors = 2}: {doors?: number} = {}) {
    this.doors = doors; }
  set doors(newDoorCount: number) {
    if (newDoorCount >= Counter.MIN_DOOR_COUNT && newDoorCount <= Counter.MAX_DOOR_COUNT) {
      this._doors = newDoorCount; }
    else { throw `Counter can only have ... `; }
  }
  get doors() {
    return this._doors; } }
\end{lstlisting}
\textbf{\textcolor{blue}{Interfaces:}}
\begin{itemize}[topsep=0pt, leftmargin=3mm]
    \setlength\itemsep{-0.3em}
    \item Interfaces (Typen) können in Deklaration von Klassen genutzt werden: Eine Klasse darf mehr als ein Interface implementieren
    \item Sie können in Deklaration von Variablen und Funktionsparametern genutzt werden: Casting ist möglich und Structural-Typing (“Duck-Typing“) wird von TypeScript unterstützt.
\end{itemize}
\begin{lstlisting}[style=htmlcssjs]
interface IPoint {
  readonly x: number; readonly y: number; }
interface ILikableItem { likes?: number; }
class DescribableItem {
  constructor(public description: string){} }
class PointOfInterest extends DescribableItem implements IPoint, ILikableItem {
  constructor(public x: number, public y: number, description: string, public likes?:number) {
    super(description); } }
// Allowed
let p: IPoint = new PointOfInterest(1, 2, "home");
let p2: PointOfInterest = p as PointOfInterest;
p2.description = "hi";
let p3: IPoint = {x: 3, y: 4}; // duck-typing
let p4: any = p3;
p4.description = "hi" // any can set anything
let p5: PointOfInterest = p3 as PointOfInterest;
p5.description = "hi";
// Not allowed
p.description; // error: Property description dies not exist on type IPoint
\end{lstlisting}


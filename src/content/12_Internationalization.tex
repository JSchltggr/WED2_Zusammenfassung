%! Author = joels
%! Date = 03/01/2021

\section{Internationalization}

% Studierende können …
% - den Unterschied von Internationalisierung, Lokalisierung und Übersetzung erklären.
% - das Konzept von «Locale» erklären.
% - wichtige Unterschiede von Sprachregionen erklären sowie welche Aspekte sich für halbautomatische Anpassung eignen und welche nicht.
% - erklären was der Einsatzbereich der folgenden Objekte ist
    % - Intl.Collator (Konstruktor)
    % - Intl.DateTimeFormat (Konstruktor)
    % - Intl.NumberFormat (Konstruktor)
    % - Intl.PluralRules (Konstruktor)
    % - Intl.ListFormat (Konstruktor)
    % - Intl.RelativeTimeFormat (Konstruktor)

\textbf{\textcolor{blue}{I18N}} - Internationalization: Programmierung, sodass Lokalisierung möglich ist\\
\textbf{\textcolor{blue}{L10N}} - Lokalisierung\\
\textbf{\textcolor{blue}{G11N}} - Globalisierung: Sprachliche und anderweitige Anpassung (häufig teil-automatischer Ersatz von Labels im UI)\\
\textbf{\textcolor{blue}{T9N}} - Translation: Übersetzung von Texten/Wörter\\
\textbf{\textcolor{blue}{Locale}} - Sprachregion: String der eine Sprachregion bestimmt\\
\textbf{\textcolor{blue}{ES2020 Internationalization Function:}}\\
\textbf{Intl:} Int globales Objekt\\
\textbf{Intl.Collator:} Sprachsensitiver Stringvergleich\\
\textbf{Intl.DateTimeFormat:} Datum/Zeit sprachsensitiv format.\\
\textbf{Intl.ListFormat:} Aufzählungen sprachsensitiv formatieren\\
\textbf{Intl.NumberFormat:} Zahlen sprachsensitiv formatieren\\
\textbf{Intl.PluralRules:} Mit Pl.sprachregeln pl.sensitiv interpolieren\\
\textbf{Intl.RelativeTimeFormat:} Relative Zeitangaben formatieren\\
\textbf{\textcolor{blue}{Lokalisierung: Was muss alles ändern?}}\\
\textbf{Automatisch:} Datum, Zeit, Zahlen, Währungen, Kalender\\
\textbf{Textübersetzung:} UI Labels, Mitteilungen, Online Hilfe\\
\textbf{Spezielle Inhalte:} Sounds, Bilder/Icons, Farben, Layout\\
\textbf{Achtung:} Masseinheit, Tel. (Zahlen) Format, Titel/Anrede, Adressformat, Seiten Layout, Lesereihenfolge, Etiquette
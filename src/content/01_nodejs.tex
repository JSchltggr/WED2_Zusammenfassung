%! Author = joels
%! Date = 03/01/2021

% Studierende können …
% - bestimmen ob JS Code der spezifische APIs nutzt für den Node Server oder für den Browser geschrieben wurde
% - die wichtigen Funktionen der folgenden Node Packages korrekt einsetzen: fs, url, http, path
    %z.B. das URL Package nutzen um aus einer Request URL die Query-String Parameter auszulesen
    %z.B. ddas API von Node http.ClientRequest und http.ClientResponse nutzen um und eine einfache Response auf einen Request (inklusive Header Info)
% - Node callback- und async-Funktionen korrekt nutzten (z.B. File API und FS-Promises API)
% - den Code eines simplen Node Web-Servers erklären bzw beobachtete Fehler korrigieren
% - .then/.catch … und await/catch bei der Verwendung von async Funktionen (und anderen Funktionen die Promises zurückgeben) korrekt nutzen.
% - callback Funktion in eine async Funktion umwandeln und umgekehrt
% - CommonJS und ES6 Syntax für die Definition und Einbindung Modulen erklären und nutzen. (inkl. Scoping von Varablen)
% - Npm package.json Informationen (Dependencies und Scripts) erklären und ensprechend Instruktionen anpassen.

\section{Node JS}
\textbf{\textcolor{blue}{Was muss ein Webserver können?}}
\begin{itemize}[topsep=0pt, leftmargin=3mm]
    \setlength\itemsep{-0.3em}
    \item HTTP Anfragen annehmen
    \item Actions ausführen basierend auf der Anfrage URL
    \item HTTP Antworten absenden
\end{itemize}
\textbf{\textcolor{blue}{Request:}} Methoden GET, PUT, POST, ...\\
\textbf{\textcolor{blue}{Response:}} Methoden writeHead, setHeader, statusCode, statusMessage, write, end
\begin{lstlisting}[style=htmlcssjs]
response.writeHead(200, {'Content-Length': body.length,'Content-Type': 'text/plain'});
response.setHeader("Content-Type", "text/html");
response.statusCode = 404;
response.statusMessage = 'Not found';
response.write("Data");
response.end("Data");
\end{lstlisting}
\textbf{\textcolor{blue}{Module:}}\\
Node verwendet für die Module Verwaltung npm.\\
\textbf{Import/Export}
\begin{lstlisting}[style=htmlcssjs]
export router; // Variable
import router from "./file.js";
export {function, otherFunction}; // several Functions
import controller from './controller.js';
export const noteService = new NoteService(); // Class
import {noteService} from "./noteServices.js";
import express from "express"; // ES6
\end{lstlisting}
\textbf{\textcolor{blue}{package.json:}}
\begin{itemize}[topsep=0pt, leftmargin=3mm]
    \setlength\itemsep{-0.3em}
    \item Beinhaltet die Informationen zum Projekt
    \item Wird benötigt um es zu publishen
    \item Wird benötigt um Module zu installieren
    \item Definiert Skripts (bsp: npm run test)
\end{itemize}

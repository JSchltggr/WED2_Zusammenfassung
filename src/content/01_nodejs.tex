%! Author = joels
%! Date = 03/01/2021

% Studierende können …
% - bestimmen ob JS Code der spezifische APIs nutzt für den Node Server oder für den Browser geschrieben wurde
% - die wichtigen Funktionen der folgenden Node Packages korrekt einsetzen: fs, url, http, path
    %z.B. das URL Package nutzen um aus einer Request URL die Query-String Parameter auszulesen
    %z.B. ddas API von Node http.ClientRequest und http.ClientResponse nutzen um und eine einfache Response auf einen Request (inklusive Header Info)
% - Node callback- und async-Funktionen korrekt nutzten (z.B. File API und FS-Promises API)
% - den Code eines simplen Node Web-Servers erklären bzw beobachtete Fehler korrigieren
% - .then/.catch … und await/catch bei der Verwendung von async Funktionen (und anderen Funktionen die Promises zurückgeben) korrekt nutzen.
% - callback Funktion in eine async Funktion umwandeln und umgekehrt
% - CommonJS und ES6 Syntax für die Definition und Einbindung Modulen erklären und nutzen. (inkl. Scoping von Varablen)
% - Npm package.json Informationen (Dependencies und Scripts) erklären und ensprechend Instruktionen anpassen.

\section{Node JS}
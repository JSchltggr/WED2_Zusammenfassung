%! Author = joels
%! Date = 03/01/2021

\section{Dev-Ops}

% Studierende können …
% - die Aufgabe und den Einsatzbereich von (CSS) Präprozessoren erklären.
% - Sass (SCSS) Code interpretieren (inkl. Variablen, Nesting, Mixing, Partials und Funktionen) und den aus Sass/SCSS Code den zugehörigen CSS Code generieren.
% - Fehler in Sass/SCSS Code identifizieren und beheben, bzw. Lücken im Code füllen.
% - mit Sass/SCSS Einheiten rechnen.
% - zwei spezifische Einsatzbereiche von PostCSS nennen.
% - erklären in welchen Kontexten es sinnvoll ist Web-Build-Tools einzusetzen und zwei spezifische Features dieser Build Tools nennen.
% - die in der Vorlesung behandelten Präprozessoren, Testing Tools, Build-Tools etc. der korrekten Kategorie zuordnen.

\textbf{\textcolor{blue}{Vorteile/Möglichkeiten von CSS Präprozessoren:}}
\begin{itemize}[topsep=0pt, leftmargin=3mm]
    \setlength\itemsep{-0.3em}
    \item Sind nicht an Limitationen von CSS gebunden
    \item Ermöglichen SE Prinzipien in CSS anzuwenden
    \item Weniger Copy Paste, Modularisierung, Wiederverwenden von Funktionalitäten
\end{itemize}
\textbf{\textcolor{blue}{Sass/SCSS Features:}}
\begin{lstlisting}[style=htmlcssjs]
// Variablen:
$purpele-navy: #635380; $main-bg-color: $purpele-navy;
body { background: $main-bg-color; }
// Verschachtelung/Nesting
nav { // auch mit > möglich, & für Eltern-Element
  ul { list-style: none; }
  li { display: inline-block; }
  a { display: block; } }
// Partials/Import separates file: _constatns.scss
@import 'constants'; // direkter Import
@use 'constants'; // definiert Namespace
// Mixins (Snippet-Wiederverwendung mittels include
@mixin visuallyhidden() { code }
.elem {@include visuallyhidden;}//mehrere include möglich
// Mixin mit Parametern
@mixin border-radius($radius: 1em) { // default Wert
    border-radius: $radius }
.box { #include border-radius(1rem); }
// Extends (Vererbung)
.icon { /* code */ } // Basisklasse
.error-icon { @extend .icon; /* specific code */ }
// Extends (Abstract Class)
%icon { /* code */ } // Keine CSS Regel für .icon
.info-icon { @extend %icon; /* code */ }
\end{lstlisting}
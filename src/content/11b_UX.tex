%! Author = joels
%! Date = 03/01/2021

\section{UX Research, Information Architecture}

% Studierende können …
% - die Bedeutung von Nielsens „1st Rule of Usability“ („don‘t Listen to Users“) erklären. Sie können zur Nützlichkeit von Fokus Gruppen beim User Research Auskunft geben.
% - den Unterschied von Problem-Space und Solution-Space erklären und warum dieser Unterschied bedeutet, dass Nutzer keine gute Quelle von Design-Vorschlägen sein sollten.
% - erklären warum bei Nutzerforschung neben der Toolverwendung auch Eigenschaften, Nutzer-Aufgaben und der Kontext dokumentiert werden sollten.
% - den Wert und die Elemente von Szenarios erklären.
% - Techniken zur Bewertung und Optimierung der Navigationsunterstützung (Fragebogen, Card-Sort, Tree-Testing) in Web-Seites erklären und sinnvoll einsetzen.
\textbf{\textcolor{blue}{Usability Kriterien nach Nielsens:}}\\
\textbf{(1)} Sichtbarkeit des System-Status \textbf{(2)} Enger Bezug zwischen System und realer Welt \textbf{(3)} Nutzerkontrolle und Freiheit \textbf{(4)} Konsistenz \& Konformität mit Standards \textbf{(5)} Fehler-Vorbeugung \textbf{(6)} Besser Sichtbarkeit als Sich-erinnern-müssen \textbf{(7)} Flexibilität und Nutzungseffizienz \textbf{(8)} Ästhetik und minimalistischer Aufbau \textbf{(9)} Nutzern helfen, Fehler zu bemerken, zu diagnostizieren und zu beheben \textbf{(10)} Hilfe und Dokumentation